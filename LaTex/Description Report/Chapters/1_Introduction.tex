\chapter{Introduction}
\section{Background}

Quantum computing stands at the forefront of technological innovation, promising to revolutionise fields such as cryptography and drug discovery. Quantum computing leverages the principles of quantum mechanics to perform computations that would not be feasible for classical computers. By utilising quantum properties such as superposition and entanglement, quantum computers have the potential to solve certain problems much faster than their classical \ccite{counterparts}{10062753}.

However, the design of effective quantum circuits remains a significant challenge, often requiring experience in both quantum mechanics and computer science. Quantum circuits, the basis of quantum algorithms, consist of a series of quantum gates applied to qubits, manipulating quantum states to perform \ccite{computations}{wong2022introduction}. Traditional approaches to quantum circuit design rely heavily on human intuition and deep knowledge, which can be an obstacle in the development of new quantum algorithms.

Platforms like IBM's \ccite{Quantum Composer}{ibmcomposer} have made it easier for researchers to manually create quantum circuits. However, as the complexity of quantum algorithms grows, manual design becomes increasingly difficult and time consuming. This limitation highlights the need for automated approaches to quantum circuit design.

\section{Aims and Objectives}

This project aims to address the challenges of manual quantum circuit creation by exploring the application of Genetic Programming (GP) to automate the design process. GP, a subset of evolutionary algorithms, offers a promising approach to automated program \ccite{synthesis}{10.1007/978-3-031-42441-0_8}, evolving computer programs by applying principles inspired by biological evolution.

The primary objectives of this project are:
\begin{itemize}
    \item Develop a complete simulation environment for quantum circuits using genetic programming
    \item Define appropriate genetic operations for circuit evolution
    \item Define fitness functions that guide the evolution process towards the desired circuit behaviour
    \item Analyse the effectiveness of evolutionary optimisation for quantum circuits by comparing them to already existing solutions using publicly available benchmarks
\end{itemize}

To ensure the quality and usability of the resulting simulation environment, it is essential to define key characteristics that must be followed:
\begin{itemize}
    \item \textbf{Simplicity}: The platform should be easily configurable for the end user, with clear parameters and intuitive interfaces.
    \item \textbf{Clear Output}: The platform should generate its output in a preferably standardised format, ensuring that users can easily read, understand, and potentially integrate the results with other tools.
    \item \textbf{Performance}: The platform should demonstrate reasonable computational efficiency.
    \item \textbf{Scalability}: The platform should be designed to run both locally on classical computers and on High Performance Computing (HPC) clusters, allowing for the evolution of more complex quantum circuits when greater computational resources are available.
\end{itemize}

\section{Studied Literature}

\nocite{*}
\printbibliography[env=nolabelbib, keyword={lit}, heading=none]

% Chapter word count: 399
