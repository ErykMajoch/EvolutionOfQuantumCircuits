\chapter{Analysis}
\section{Problem Decomposition}

The task of optimising automated quantum circuit design can be decomposed into three main categories:
\begin{enumerate}
    \item \textbf{Genetic Programming Algorithm}
    \item \textbf{Information Representation}
    \item \textbf{Simulation Program}
\end{enumerate}

Each of these categories presents unique challenges that need to be addressed to achieve the desired outcome. All the identified challenges also contain possible solutions which will be considered in the final solution and explained in a later section of the dissertation. They'll be experimented on or combined in order to see which yield the best possible outcomes. This is only an initial proposal of possible solutions as more may arise through further research if generated results aren't of expected quality.

\subsection{Genetic Programming Algorithm}
Genetic programming frameworks allow us to control every step of the evolution procedure. With this, we are able to tailor the framework to our needs. Here are the key aspects which will be fine-tuned as a part of this projects research:

\begin{table}[h!]
    \centering
    \setlength\tabcolsep{0pt}
    \begin{tabular*}{\linewidth}{@{\extracolsep{\fill}}|c|c|c|c|} 
        \hline
        \textbf{Fitness Functions} & \textbf{Selection Methods} & \textbf{Crossover Methods} & \textbf{Mutation Methods} \\ \hline
        Fidelity Function & Random Selection & Single-point Crossover & Single Gate Flip \\ [1ex]
        Entanglement Function & Tournament Selection & Multi-point Crossover & Mutate Qubits \\ [1ex]
         &  &  & Mutate Gates \\ [1ex]
        \hline
    \end{tabular*}
    \caption{Genetic Algorithm Decomposition Table}
\end{table}


\subsection{Information Representation}
The next significant problem concerns the representation of any data stored within the program. There needs to be a way for the user to provide their circuit requirements in a simple way when using the program. During the evolutionary process, the circuits must be represented in such way that the algorithm may perform its steps comfortably. When simulation has finished, it would be preferable to use a standardised format whilst exporting the generated circuit in order to keep the possibility of using them with other tools open.

\begin{table}[h!]
    \centering
    \begin{tabular}{|c|c|c|c|} 
        \hline
        \textbf{User's Description} & \textbf{Circuit Representation} & \textbf{Output Generation} \\ \hline
        Unitary Matrix & String Encoding & Visual Circuit Diagram \\ [1ex]
        Truth Table & Directed Acyclic Graph & Qiskit Class Instance \\ [1ex]
        Circuit Properties & Quantum Assembly Language & \\ [1ex]
        \hline
    \end{tabular}
    \caption{Information Representation Decomposition Table}
\end{table}

\subsection{Simulation Program}
The final significant consideration is the user's interaction with the program itself. The simulation environment needs to be versatile by allowing the user to configure all of their requirements as well as be user-friendly.

\begin{table}[h!]
    \centering
    \begin{tabular}{|c|c|c|c|} 
        \hline
        \textbf{Interface} & \textbf{Scalability} \\ \hline
        Command Line Interface (CLI) & Local execution on personal computers \\ [1ex]
        Graphical User Interface (GUI) & \makecell{Distributed execution on\\High Performance Computing (HPC) clusters} \\ [1ex]
        \hline
    \end{tabular}
    \caption{Simulation Program Decomposition Table}
\end{table}

\section{Proposed Technologies}
To address these challenges I will be using the following technologies. The program will be written using the Python programming \ccite{language}{python} due to its popularity in machine learning tasks as well as extenisve availability of third-party frameworks. The genetic programming framework of choice is \ccite{PyGAD}{gad2023pygad} due to its customisability and for the GUI I've decided on \ccite{PyQt}{pyqt} due to my familiarity with its C++ library. I will also be using IBM's Qiskit quantum computing \ccite{framework}{qiskit} which is known for its representation, simulation and visualisation of quantum circuits. 

% Chapter word count: 359
